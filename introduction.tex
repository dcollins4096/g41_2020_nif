\section{Introduction}

\noindent \textbf{Opening paragraph}

Stars form in clouds of molecular hydrogen that are large (hundreds of light
years), low density ($\rho \simeq 10^{-24}\rm{g}\ \rm{cm}^{-3}$) ,
cold ($T \simeq 20 K$), and turbulent, with velocities several times the speed
of sound, $c_s$, in the
cloud ($c_s\simeq 0.2 \rm{km}\ \rm{s^{-1}}$).  It is an essential but open problem in astrophysics to uncover the relationship between
the 
properties of a molecular cloud (e.g. density and velocity distributions)
and the properties of the stellar population it produces (e.g.  star formation
rate and mass distribution).  The current proposal
will measure the relationship between the velocity of a supersonic shock and the
distribution of the resulting density fluctuations induced by the post-shock
turbulence, $f(\rho)$.  This is an essential component
in understanding the relation between the velocity distribution in a molecular
cloud, which is easy to measure with telescopes, and its potential to form stars, which impacts
many aspects of the host galaxy.\citep{Padoan02}

\noindent \textbf{Scientific Motivation} 
Molecular clouds are barely held together by gravity. Their gravitational
energy, which will cause the cloud to collapse, is balanced
by kinetic, magnetic, and thermal energies, which all support against the
collapse.  
Interacting shocks within the cloud cause density enhancements orders of
magnitude larger than the mean density of the cloud.  Some fraction
of these small, dense regions can have large enough gravitational binding
energies
to decouple from the turbulence,  and ultimately become stars.  The proposed
campaign will examine these post-shock density enhancements.

The fraction
of a cloud that can form stars $(M_*/M_{total})$ can be predicted from the density distribution,
$f(\rho)$,  as
the mass fraction of the cloud 
above a critical density, $\rho_c$;
\begin{align}
%M_*/M_{total} = \int_{\rho_c}^\inf f(\rho) d\rho.\label{eqn_mass}
M_*/M_{total} = \int_{\rho_c}^\infty \rho f(\rho) d\rho. \label{eqn_mass}
\end{align}
Similar
calculations can give predictions for the star formation rate and mass
distribution of the stellar population. The critical density $\rho_c$ is the density at which gravity
overtakes the other forces.  
With this campaign, we will measure the full density distribution, $f(\rho)$, by
way of the integrated column density distribution using the techniques of (cite
Federrath). Since $f(\rho)$ is set primarily by the underlying turbulence, our
proposed campaign will give a direct window on the relationship between the
properties of a star forming cloud and the stellar population it may form.

It is straightforward to show, analytically and numerically, that supersonic,
isothermal turbulence will produce a lognormal density distribution
$f(\rho)$;
\begin{align}
f(\rho) = \frac{1}{\rho \sqrt{2 \pi \sigma_{\ln \rho}^2}} e^{-\frac{(\ln \rho-\mu)^2}{2
\sigma_{\ln \rho}^2}}.\label{eqn_frho}
\end{align}
Here $\mu$  and $\sigma_{\ln \rho}$ are the mean and variance of $\ln \rho$. One very useful feature of this distribution is that the width of
the distribution, $\sigma_{\ln \rho}$, is simply determined by the velocity variance
\begin{align}
\sigma_{\ln \rho}^2 = \ln\left( 1 + b (\sigma_v/c_s)^2\right).\label{eqn_sigma},
\end{align}
where $\sigma_v$ is the velocity variance and $c_s$ the speed of sound.
This can be extended to gas that is adiabatic or with external heating and
cooling sources, which tends to result in a similar distribution, but with excess
tails to higher or lower density, depending on the compressibility of the gas
relative to isothermal.  The utility of this relationship, which is well
established numerically \emph{citations}, is that it predicts the density
distribution only using the velocity variance.  This has two primary
consequences.  First, the velocity variance is relatively easy to measure with
radio telescopes, using Doppler shift of emission lines.  Second, it makes
prediction of the formation of stars as straight forward as the application of
Equations \ref{eqn_sigma}, \ref{eqn_frho}, and \ref{eqn_mass}.  
Our proposed campaign will examine and verify the validity of  Equations
\ref{eqn_frho} and \ref{eqn_sigma}.
%

Current theoretical and
numerical evidence indicate that, for fully developed supersonic turbulence, the density variance is related to the velocity
variance as
\begin{align}
\sigma_\rho^2 = b (\sigma_v/c_s)^2.\label{eqn_sigma}
\end{align}
The environment of the clouds birth determine the velocity variance, $\sigma_v$,
which then determines the density variance, $\sigma_\rho$.  The parameter $b$
is of order unity.  

A combined with the other forces in the
gas, then determines the fraction of the cloud that is
dense enough to collapse.  Equation \ref{eqn_sigma} can be reasoned by
order-of-magnitude arguments, and seems to stand up to simulations.  However, to
date this has not been experimentally verified.






